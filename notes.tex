

The Paley-Wiener space of finite-energy bandlimited signals is an RKHS. This space, notably, excludes many bandlimited signals used in signal-processing, such as the (infinite-energy) modulation tone, whose Fourier transform is a tempered distribution that falls within the bandlimit but that is not an element of $L^2(-\pi, \pi)$.

Orthogonal projection from an RKHS $\mathcal{H}$ to a (necessarily closed, convex) subspace $\mathcal{H}_0 \subsetneq \mathcal{H}$ certainly maps RKHSs to RKHSs and, indeed, gives us an easy way to find the kernel of $\mathcal{H}_0$: for any $x\in\mathcal{X}$, the representation of evaluation $k_x\in\mathcal{H}$ can be written $k_x = \mathrm{P}_{\mathcal{H}_0}k_x + \mathrm{P}_{\mathcal{H}_0^\perp}k_x$. For any $x\in\mathcal{X}$ and $f\in\mathcal{H}_0$, we have that $f = \mathrm{P}_{\mathcal{H}_0}f$, and 
\begin{align*}
f(x) &= \langle f, k_x\rangle_{\mathcal{H}} = \left\langle \mathrm{P}_{\mathcal{H}_0} f,  k_x\right\rangle_{\mathcal{H}} =  \left\langle \mathrm{P}_{\mathcal{H}_0}^2 f,  k_x\right\rangle_{\mathcal{H}} \\ 
&=  \left\langle \mathrm{P}_{\mathcal{H}_0} f,  \mathrm{P}_{\mathcal{H}_0} k_x\right\rangle_{\mathcal{H}} =  \left\langle \mathrm{P}_{\mathcal{H}_0} f,  \mathrm{P}_{\mathcal{H}_0} k_x\right\rangle_{\mathcal{H}_0} =  \left\langle f,  \mathrm{P}_{\mathcal{H}_0} k_x\right\rangle_{\mathcal{H}_0},
\end{align*}
so that  $\mathrm{P}_{\mathcal{H}_0} k_x$ is the Riesz representation of evaluation at $x$.

On the other hand, the canonical between $L^2(\mathbb{R})$ to $\ell^2$ (i.e., the analysis map from a function to its Fourier coefficients on, say, a Hermite basis, and the synthesis adjoint) do not preserve RKHSness. 

Since there is a unique kernel 

Are there any results on conditions for operators that preserve RKHSness? That is a map from an RKHS $\mathcal{H}$ of functions on some index set $\mathcal{X}$ to the space of functions $\mathcal{H}'$ on some other index set $\mathcal{X}'$ such that for all $x\in\mathcal{X}'$, there is an evaluation functional at $x$ in the continuous dual of $\mathcal{H}'$ i.e. there is a representation of evaluation $k_x$
$\forall f\in\mathcal{H}', \forall x\in\mathcal{X}', \; f(x) = \langle f, k_x\rangle  

While $L^2(\mathbb{T})$ is not an RKHS, we can impose constraints on the elements of the space, such as a finite wiggliness penalty (as is the case for Wahba's smoothing splines), to create RKHSs of functions on $\mathbb{T}$ of infinite dimension. The bandlimitedness constraint is, in this context, a blunt instrume nt: the space of bandlimited functions on the circle $\mathbb{T}$ is finite-dimensional and thus an RKHS.


I do not care whether or not the index set changes. If the index set $\mathcal{X}$ is the same, let $\mathrm{T}:\mathcal{H}\to\mathcal{H}'$ and $k$ be the unique reproducing kernel associated with $\mathcal{H}$, $k'$ the unique reproducing kernel with $\mathcal{H}'$. We write our Riesz representations of evaluation as follows $\forall (x,y)\in\mathcal{X}^2, k(x,y) = \langle k_x, k_y\rangle_{\mathcal{H}}$ with $k_x \overset{\text{def}}=k(\cdot,x)$. Then for each $x\in\mathcal{X}$, we have that $\forall f\in\mathcal{H}\cap\mathcal{H}'$
$f(x) = \langle f, k_x\rangle_{\mathcal{H}}$ and $(Tf)(x) = \langle Tf, k'_x\rangle_{\mathcal{H}}$ 

